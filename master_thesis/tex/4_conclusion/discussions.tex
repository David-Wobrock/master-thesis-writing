\paragraph{Linear solvers \& domain decomposition methods on dense matrices}
As we seen through the experiments, the resolution of linear systems scales slightly with respect to the number of processors.
Domain decomposition methods improve the performances of solving dense systems of linear equations, even without overlap.
In our small case, a direct solver showed the best results for preconditioning.
This might not be the case for larger systems.

\paragraph{Gram-Schmidt process}
We saw that the orthogonalisation process is difficult to parallelise efficiently.
Skipping the Gram-Schmidt procedure every other iteration, to stabilise the algorithm less often, gave an improvement, but we cannot totally avoid the cost of it when increasing the number of processors.
This problem is well-known and one of the biggest limitations for scaling diverse algorithms to a large number of processors.

The Gram-Schmidt procedure orthogonalises a set of vectors by sequentially substracting from a vector the projections on the previously orthogonalised vectors.
The inner product of two vectors is computed frequently, because of the projection, and since each vector is shared over all processors, a lot of communication is involved in this operation.
Attempts for parallel implementation are numerous, like \cite{katagiri_parallel_gram_schmidt_2003}, but they either still have many communications or they suggest a different memory distribution schema.

% Even if more skipping, improvement limited ???
