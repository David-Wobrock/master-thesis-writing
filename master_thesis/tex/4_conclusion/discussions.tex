\paragraph{Linear solvers \& domain decomposition methods on dense matrices}
The presented experiments show that the resolution of linear systems scales with respect to the number of processors.
Domain decomposition methods improve the performances of solving dense systems of linear equations, even without overlap, in the context of image processing.
Indeed, these methods are naturally parallel and precondition the matrix appropriately.
This becomes clear when observing the performances improvement on large matrices.

In the case of a small input image, a direct solver shows the best results for computing the inverses for preconditioning.
On larger inputs, these are not available but using iterative Krylov type methods on the subdomains also exposes a considerable improvement compared to the case without preconditioner.

\paragraph{Gram-Schmidt process}
The orthogonalisation process is difficult to parallelise efficiently.
Skipping the Gram-Schmidt procedure every other iteration, to stabilise the algorithm less often, gave an improvement, but we cannot totally avoid the cost of it when increasing the number of processors.
Skipping the Gram-Schmidt procedure more often tends to improve the runtime when a large amount of processors is involved.
However, the stability of the algorithm diminishes when omitting numerous orthogonalisations.
This scalability problem is well-known and one of the biggest limitations for scaling diverse algorithms to a large number of processors.

The Gram-Schmidt procedure orthogonalises a set of vectors by sequentially substracting from a vector the projections on the previously orthogonalised vectors.
The inner product of two vectors is computed frequently, because of the projection, and since each vector is shared over all processors, a lot of communication is involved in this operation.
Attempts for parallel implementation are numerous, like \cite{katagiri_parallel_gram_schmidt_2003}, but they either still have many communications or suggest a different memory distribution schema.
