Even if we cannot approximate the trailing eigenvectors of \(\Lapl\) through the eigenvectors of \(\Lapl_A\), we still define how the Laplacian is used to compute the output image.
The summary \cite{modern_tour_2013} implicitly defines the filter as \(W = I - f(\Lapl)\) with the function \(f\) that helps achieving various filters.
To apply the function efficiently to the Laplacian operator, we apply it to the diagonal eigenvalue matrix such as \(f(\Lapl) = \Phi f(\Pi) \Phi^T\).
The output image using the filter approximation \(\tilde{W}\) can be expressed as:
\begin{equation}
 \begin{split}
     \tilde{z} & = \tilde{W}y \\
               & = (I - f(\tilde{\Lapl})) y \\
               & = (I - \tilde{\Phi} f(\tilde{\Pi}) \tilde{\Phi^T}) y \\
               & = y - \tilde{\Phi} f(\tilde{\Pi}) \tilde{\Phi^T} y
 \end{split}
\end{equation}

\begin{algorithm}[H]
 \caption{Image processing using approximated graph Laplacian operator}
 \begin{algorithmic}
  \REQUIRE \(y\) an image of size \(N\), \(f\) the function applied to \(\Lapl\)
  \ENSURE \(\tilde{z}\) the output image by the approximated filter
  \STATE \COMMENT{Sampling}
  \STATE Sample \(p\) pixels, \(p \ll N\)
  \STATE \COMMENT{Kernel matrix approximation}
  \STATE Compute \(K_A\) (size \(p \times p\)) and \(K_B\) (size \(p \times (N-p)\))
  \STATE Compute the Laplacian submatrices \(\Lapl_A\) and \(\Lapl_B\)
  \STATE \COMMENT{Eigendecomposition}
  \STATE Compute the \(m\) smallest eigenvalues \(\Pi_A\) and the associated eigenvectors \(\Phi_A\) of \(\Lapl_A\)
  \STATE \COMMENT{Nystr\"om extension and compute the filter}
  \STATE See methods of solution proposed by \cite{fowlkes_spectral_2004}
  \STATE \(\tilde{z} \leftarrow \tilde{W} y\)
 \end{algorithmic}
\end{algorithm}
