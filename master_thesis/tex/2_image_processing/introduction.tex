\paragraph{}
A global image filter consists of a function which outputs one pixel, taking all pixels as input and applying weights to them.
Let \(z_i\) being the output pixel, \(W_{ij}\) the weight, \(N\) the number of pixels in the image and \(y_j\) all input pixels.
We compute an output pixel with:
\[z_i = \sum^{N}_{j=1} W_{ij}y_j,\]
This means that a vector of weights exists for each pixel.

As a practical notation, we can say that, with \(W\) the matrix of weights and \(y\) and \(z\) respectively the input and output images as vectors,
\[z = Wy.\]
The filter matrix \(W\) considered here is data-dependent and built upon the input image \(y\).
A more mathematical notation would consider \(W\) as a nonlinear function of the input image such as \(z = W(y) y\).
