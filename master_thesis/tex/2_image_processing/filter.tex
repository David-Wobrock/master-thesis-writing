Multiple graph Laplacian definitions, more or less equivalent, exist and can have slightly different properties.
\ifthesis
The table \ref{table:laplacians} on page \pageref{table:laplacians} proposes a summary of most Laplacian matrix definitions.
\fi
In the case of image smoothing, the filter \(W\) is roughly defined such as \(\Lapl = I - W\) \cite{siam_slides_2016} and so \(W = I - \Lapl\).
To get various filters using one Laplacian, we can apply some function \(f\) to \(\Lapl\) and obtain \(W = I - f(\Lapl)\) which gives us more possibilities on the filter computation.
\ifthesis
 Below the global filter algorithm if we compute the entire matrices:

 \begin{algorithm}[H]
  \caption{Image processing using entire graph Laplacian operator}
  \begin{algorithmic}
   \REQUIRE \(y\) an image of size \(N\), \(f\) the function applied to \(\Lapl\)
   \ENSURE \(z\) the output image
   \STATE Compute \(K\) (size \(N \times N\))
   \STATE Compute Laplacian matrix \(\Lapl\)
   \STATE \(z \leftarrow (I - f(\Lapl)) y\)
  \end{algorithmic}
 \end{algorithm}
\fi

However, all these matrices \(K\), \(\Lapl\) and \(W\) represent huge computational costs.
Only storing one of these matrices is already a challenge since they have a size of \(N^2\).

For example, a tiny test image of size \(256 \times 256\) has 65 536 pixels, so one of these matrices has approximately \(4.29 \times 10^9\) elements.
Considering storing those with a 64 bits type, one matrix takes more than 34 GB of memory.
Scaling to a modern smartphone picture, taken with a 10 megapixel camera, a matrix contains \(10^{14}\) elements, meaning 800 TB of memory for each matrix.
