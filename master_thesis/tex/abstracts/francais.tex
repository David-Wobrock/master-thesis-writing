Les plus récentes méthodes de traitement d'images sont basées sur des filtres globaux dépendant des données.
Ces méthodes impliquent d'immenses matrices de similarités et de Laplacien de graphe de l'image en entrée.
Puisque ces matrices ne tiennent pas en mémoire, il est nécessaire de les approcher par l'échantillonnage et l'extension de Nystr\"om basée sur la décomposition spectral des matrices.
Le problème aux valeurs propres qui en découle concerne les plus grandes valeurs propres du filtre.
Néanmoins, celle-ci corresponde aux plus petites valeurs propres du Laplacien.
Pour résoudre ce problème, nous utilisons la méthode la puissance inverse, qui converge mieux dans ce cas, mais qui implique la résolution de systèmes linéaires.
Dans cette thèse de master, nous étudions le comportement de la résolution de systèmes d'équations linéaires sur de grandes matrices pleines, dans le cadre du traitement d'images, en utilisant des méthodes de décomposition de domaines comme préconditionneur.
Les expériences montrent que les méthodes de type Krylov combinées avec les méthodes de Schwarz comme préconditionneur passe bien à l'échelle pour de grandes systèmes linéaires avec des matrices pleines.
