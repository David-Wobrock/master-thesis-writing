De senaste bildbehandlingsmetoderna är baserade på global databeroende filter.
Dessa metoder innefattar stora affinitetsmatriser som inte kan passa i minnet och måste approximeras med spektral sönderdelning.
Det härledda egenvärdesproblemet gäller de minsta egenvärdena för den laplaciska operatören som kan lösas med hjälp av den inverse iterationsmetoden som i sin tur innebär att lösa linjära system.
I denna masterprojekt beskriver vi hur spektralalgoritmen fungerar för bildredigering och utforskar beteendet att lösa stora och täta system av linjära ekvationer parallellt i samband med bildbehandlingVi använder Krylov-typlösare, till exempel GMRES, och förutsätter att dom använder sönderdelningsmetoder för att lösa systemen på högpresterande kluster.
Experimenten visar att Schwarz-metoderna som preconditioner skala liksom vi ökar antalet processorer på stora bilder.
Vi observerar emellertid att den begränsande faktorn är Gram-Schmidt-ortogonaliseringsproceduren.
Vi jämför också prestationerna till den toppmoderna Krylov-Schur-algoritmen tillhandahållet av SLEPc-biblioteket.
