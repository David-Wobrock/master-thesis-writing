De senaste bildbehandlingsmetoderna är baserade på globala databeroende filter.
Dessa metoder innefattar stora affinitetsmatriser som inte får plats i minnet och måste approximeras med spektral sönderdelning.
Det härledda egenvärdesproblemet gäller de minsta egenvärdena för den laplaciska operatorn som kan lösas med hjälp av den inverserade iterationsmetoden som i sin tur innebär att lösa linjära system.
I detta examensarbete beskriver vi hur spektralalgoritmen fungerar för bildredigering och utforskar dess beteende att lösa stora och täta system av linjära ekvationer parallellt i samband med bildbehandling.
Vi använder Krylov-typlösare, till exempel GMRES, och förutsätter att de använder sönderdelningsmetoder för att lösa systemen på högpresterande kluster.
Experimenten visar att användning av Schwarz-metoderna som förbehandling har en bra skalbarhet då vi ökar antalet processorer på stora bilder.
Vi observerar emellertid att den begränsande faktorn är Gram-Schmidt-ortogonaliseringsproceduren.
Vi jämför också prestationerna med den toppmoderna Krylov-Schur-algoritmen tillhandahållen av SLEPc-biblioteket.
