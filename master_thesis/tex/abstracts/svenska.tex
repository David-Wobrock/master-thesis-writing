De senaste bildbehandlingsmetoderna är baserade på global databeroende filter.
Dessa metoder innefattar stora affinitetsmatriser och grafen Laplaceoperatören av inmatningsbild.
Eftersom dessa matriser inte kan passa i minnet, approximeras de genom provtagning och användning av Nyströmförlängningen baserat på matrisens spektral sönderdelning.
Det härledda egenvärdesproblemets gäller de största egenvärdena för filtret.
De motsvarar emellertid de Laplaceoperatörens minsta egenvärden.
För att lösa detta använder vi den inverse iterationsmetoden, som har en bättre konvergenshastighet, men innebär att lösa linjära system.
I den här examensarbeten studerar vi beteendet för att lösa stora och täta linjära ekvationssystem, i samband med bildbehandling, med användning av domänavbrytningsmetoder som preconditioner.
Experimenten visar att Krylov-typmetoder kombinerat med Schwarz-metoder som preconditioner skala väl med avseende på stora och täta linjära system.
