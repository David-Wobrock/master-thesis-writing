Fundamentally, the orthogonalisation is used to stabilise the algorithm.
To accelerate our algorithm further, we try to orthogonalise the vectors \(X_k\) every other iteration instead of every iteration.
We present below the resulting performances:

\begin{figure}[H]
  \centering
  \begin{tikzpicture}
 \begin{groupplot}[
  group style={
   group size=2 by 1,
   xlabels at=edge bottom,
   ylabels at=edge left,
   horizontal sep=1.5cm,
   },
  height=6cm,
  width=0.5\textwidth,
  xlabel=Number of processors,
  xtick={2, 4, 8, 16, 32, 64, 192},
  xmode=log,
  log ticks with fixed point,
  ylabel=Runtime (seconds)
  ]
  \nextgroupplot[title={50 eigenvalues.}]
   \addplot[color=blue, mark=x] coordinates {
    (2, 819.1718588)
    (4, 404.5779052)
    (8, 207.887785)
    (16, 109.2656354)
    (32, 61.5224218)
    (64, 35.5858806)
    (96, 28.0760846)
    (128, 24.3417164)
    (160, 22.365463)
    (192, 21.1874868)
   };

  \nextgroupplot[title={500 eigenvalues.}]
   \addplot[color=blue, mark=x] coordinates {
    (2, 5613.950337)
    (4, 3614.57139)
    (8, 2804.417351)
    (16, 2346.925115)
    (32, 2223.999989)
    (64, 2104.544023)
    (96, 2148.624336)
    (128, 2264.674778)
    (160, 2226.515463)
    (192, 2494.121987)
   };
 \end{groupplot}
\end{tikzpicture}

  \caption{Runtime of the inverse subspace iteration with skipping the Gram-Schmidt procedure every other iteration (log scale).}
\end{figure}

The performances for 50 eigenvalues are similar to the case when we are not skipping the Gram-Schmidt orthogonalisation every other iteration.
We saw that only a small proportion of time is spent doing the orthogonalisation in this case, so the impact is not significant.

However, for computing 500 eigenvalues, the runtime with skipping the orthogonalisation every other iteration is much lower.
Most time is spent doing the Gram-Schmidt process, so the execution is considerably sped up.
For 2 processors, the runtime is around 5600 seconds and the fastest runtime is 2100 seconds for 64 processors.
Even if the algorithm requires a few more outer iterations to converge, we nearly observe a factor 2 speed up with respect to the algorithm without skipping the Gram-Schmidt procedure.
The communication overhead of the method remains a problem when the number of processors is large.

When skipping the Gram-Schmidt more often than every other iteration, we might see further improvements.
Below the runtime for 2 and 64 processors of the inverse subspace algorithm depending on the frequency of orthogonalisation:

\begin{figure}[H]
  \centering
  \begin{tikzpicture}
 \begin{groupplot}[
  group style={
   group size=2 by 1,
   xlabels at=edge bottom,
   ylabels at=edge left,
   horizontal sep=2.2cm,
   },
  height=6cm,
  width=0.5\textwidth,
  xlabel={Orthogonalisation every \(x\) iterations},
  xtick={1, 2, 3, 4, 5},
  %xmode=log,
  %ymode=log,
  %log ticks with fixed point,
  ylabel=Runtime (seconds),
  legend style={
   at={(1, -0.3)},
   anchor=north
  },
  ]
  \nextgroupplot[title={2 processors, 500 eigenvalues.}]
   \addplot[color=blue, mark=x] coordinates {
    (1, 7149.885403)
    (2, 5613.950337)
    (3, 0)
    (4, 0)
    (5, 0)
   };
   \addlegendentry{Runtime}
   \addplot[color=black, domain=1:5, dashed] expression {
    7149.885403/x};
   \addlegendentry{Linear speedup}

  \nextgroupplot[title={64 processors, 500 eigenvalues.}]
   \addplot[color=blue, mark=x] coordinates {
    (1, 3712.591258)
    (2, 2104.544023)
    (3, 0)
    (4, 0)
    (5, 0)
   };
   \addplot[color=black, domain=1:5, dashed] expression {
    3712.591258/x};
 \end{groupplot}
\end{tikzpicture}


  \caption{Runtime of the inverse subspace iteration depending on the amount of Gram-Schmidt procedures.}
\end{figure}

We see that...
