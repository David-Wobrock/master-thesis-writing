\paragraph{}
To scale our algorithm to use usual camera pictures, but also much larger inputs, we implemented it in a parallel manner using the C language and the Portable, Extensible Toolkit for Scientific Computation (PETSc) \cite{petsc_web_page}.
This library is built upon MPI and contains distributed data structures and parallel scientific computation routines.
The most useful are the matrix and vector data structures and the parallel matrix-matrix and matrix-vector products.
Additionally, PETSc provides Krylov subspace methods and preconditioners for solving linear systems, also implemented in a scalable and parallel manner.
In a nutshell, PETSc provides an impressive parallel linear algebra toolkit which is very useful to shorten the development time.
As we are basically using MPI, the main parallelism technique that we apply is SPMD.
It is possible to activate some SIMD parallelism with PETSc but we do not consider it in our case.
We want to point out to the reader that the distributed PETSc matrix data structure splits the data without overlap in a row-wise distribution manner.

In order to verify the correctness of our implementation, we used the Scalable Library for Eigenvalue Problem Computation (SLEPc) \cite{hernandez_slepc_2005}, which is based on PETSc and provides parallel eigenvalue problem solvers.
Furthermore, we need the library Elemental \cite{poulson_elemental_2013} in order to achieve dense matrix operations in PETSc.

We present how we included parallelism in our algorithm step-by-step, starting with reading the image and sampling.
Then follows the computation of the affinities of the sampled pixels.
And we finish with the computation of the smallest eigenvalues using the inverse subspace iteration.
The implementation associated to this project is open source and can be found on GitHub\footnote{\url{https://github.com/David-Wobrock/image-processing-graph-laplacian/}}.
