The experiments are done on the test cluster of the Laboratory Jacques-Louis Lions at Sorbonne University (formerly University Pierre and Marie Curie).
This computer has 32 CPUs of 10 cores each, clocked at 2.4 GHz, and a total memory of 2 TB.
The setup of the experiments consists of running a specific test with different parameters, scaling the algorithm up to 192 processors.
The code is compiled on this computer using GCC 6.3.0, without compiler flags, and the MPI implementation is Open MPI 1.8.3.
The versions of other libraries are PETSc 3.8.3, SLEPc 3.8.2 and Elemental 0.87.7.

To get the results, we run an algorithm multiple times for each number of processors, usually from 2 up to 192 processors, and average the runs to insure a certain accuracy.
Over the experiments, we noticed some stability of the runtimes, so that the standard deviation of multiple runs always remained somehow small.
Therefore, we do not show the standard deviation on the figures.

Also, the plots show the theoretical linear speedup that represents perfect scalability, where doubling the number of processors halves the runtime.
