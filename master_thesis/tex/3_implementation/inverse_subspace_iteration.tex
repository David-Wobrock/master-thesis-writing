The used algorithm to compute the smallest eigenvalues is the inverse subspace iteration inspired by \cite{el_khoury_acceleration_2014}.
With \(m\) the number of eigenvalues we will compute, \(p\) the sample size and \(m \le p\), we start the algorithm by selecting \(m\) random orthonormal independent vectors \(X_0\) of size \(p\).

The inverse iteration algorithm consists of outer and inner iterations, with \(k\) the index of the current outer iteration.
The inner iteration consists of solving \(m\) linear systems, one for each vector of \(X_k\) that we approximate, such that \(\forall i \in [1, m]\) and \(X_k^{(i)}\) the \(i\)th vector of the subspace \(X_k\):
\[A X_{k+1}^{(i)} = X_k^{(i)}.\]
The outer iteration consists of repeating this process and orthonormalising the new vectors \(X_{k+1}\) until convergence, meaning having a small enough residual norm.
We define the residual \(R_k\) of \(X_k\), at a certain iteration \(k\), as
\begin{equation}
 \begin{split}
  R_k & = A X_k - X_k X_k^T A X_k \\
      & = (I - X_k X_k^T) A X_k.
 \end{split}
\end{equation}

We implemented a parallel Gram-Schmidt orthonormalising routine, based on the classical sequential one.
A summary of the inverse subspace iteration algorithm:

\begin{algorithm}[H]
 \caption{Inverse subspace iteration}
 \begin{algorithmic}
  \REQUIRE \(A\) the matrix of size \(p \times p\), \(m\) the number of required eigenvalues, \(\varepsilon\) required precision of the subspace
  \ENSURE \(X_k\) the desired invariant subspace
  \STATE Initialise \(m\) random orthonormal vectors \(X_0\) of size \(p\)
  \STATE For k=0, 1, 2, \dots
  \WHILE{\(\|R_k\| > \varepsilon\)}
   \FOR{i=1 \TO m}
    \STATE Solve \(A X_{k+1}^{(i)} = X_k^{(i)}\)
   \ENDFOR
   \STATE Orthonormalise \(X_{k+1}\)
  \ENDWHILE
 \end{algorithmic}
\end{algorithm}

Solving the systems of linear equations is done using the Krylov type solvers and the preconditioners included in PETSc.
As a standard approach, we use the Restricted Additive Schwarz (RAS) method as preconditioning method, without overlap and 2 domains per process.
Each subdomain is solved using the GMRES method.

On each outer iteration, we must compute the residuals to see if we converged.
This requires multiple matrix-matrix products and computing a norm, so communication cannot be avoided here.
