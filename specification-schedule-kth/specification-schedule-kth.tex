% Template inspired from https://github.com/hleskela/kth-master-thesis-templates/

\documentclass[12pt]{article}

% This is the preamble, load any packages you're going to use here
\usepackage{enumerate} % allows us to customize our lists
\usepackage{pgfgantt} % for gantt chart
\usepackage{geometry} % to change margins
\usepackage{ragged2e} % provides \RaggedLeft
\usepackage{hyperref} % provides \href

\begin{document}

\title{Specification and schedule}
\author{David Wobrock\ \texttt{wobrock@kth.se}}

\maketitle

\section*{Formalities}

\begin{itemize}
 \item Preliminary title: Spectral Graph Theory and High Performance Computing
 \item CSC Supervisor: Stefano Markidis (\texttt{markidis@kth.se})
 \item CSC Examiner: Erwin Laure (\texttt{erwinl@pdc.kth.se})
 \item Supervisor at principal's workplace: Fr\'{e}d\'{e}ric Nataf (\texttt{nataf@ann.jussieu.fr})
\end{itemize}

\section*{Background \& Objective}

\subsection*{Description of the area within which the degree project is being carried out}

Spectral Graph Theory is a developing field with many applications through sociology, chemistry, biology, electrical engineering, computer science, topology and social networks such as Facebook or LinkedIn.
Network science studies the mathematical structure of the graphs that arise in these diverse fields, and the design, analysis, and applications of algorithms that compute with and on them.
For now days large graphs as the ones from social networks, memory requirements and computational times have become an issue.

\subsection*{The principal's interest}

Explore more possibilities of spectral graph theory, involving eventually parallel programming.

\subsection*{Objective}

The objective of this project is to compare preconditioning techniques used for linear systems arising in graph theory.
An emphasis will be put on domain decomposition methods which are natural parallel hybrid solvers and which have been seldom used yet for these problems.
These methods are naturally parallel and adapted to distributed data storage.

\section*{Research Question \& Method}

\subsection*{The question that will be examined}

%TODO Formulated as an explicit and evaluable question.

\subsection*{Specified problem definition}

%TODO E.g. what does the assignment entail and what are the challenges involved?

\subsection*{Examination method}

% TODO Preliminary description of, for example, algorithms that will be tested, data that will be used.

\subsection*{Expected scientific results}

%TODO How is the work scientifically relevant and what is the hypothesis being tested? How is this hypothesis being tested?

\section*{Evaluation \& News Value}

\subsection*{Evaluation}

%TODO How is it determined if the objective of the degree project has been fulfilled and if the question has been adequately answered? Preliminary report on the evaluation method, measures and data.

\subsection*{The work's innovation/news value}

% TODO Why does someone want to read the finished work? And who are these people?

\section*{Pilot Study}

\subsection*{Description of the literature studies}

%TODO What areas will the literature study focus on? How shall the necessary knowledge on background and state-of-the-art be obtained? What preliminarily important references have been identified?

\section*{Conditions \& Schedule}
% TODO List of the resources expected to be needed to solve the problem (unless the degree project involves investigating what equipment should be used). This can be technical equipment, but also experiment and interview subjects. For instance:

% TODO
\begin{itemize}
 \item hardware, my laptop
 \item software, R and latex
 \item people, test subjects and interviewees
 \item perseverance, because it's tough to write a thesis
\end{itemize}

\subsection*{Defined limitations on what is to be done}

% TODO So that it is clear what is not included in the degree project

\subsection*{Collaboration with the principal}

% TODO Describe the way in which the principal will be involved in the project and what the external supervisor has undertaken to do (e.g. in terms of discussion, implementation, report reading).

\newpage
\section*{Schedule}

% TODO

\end{document}
