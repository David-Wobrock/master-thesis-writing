% Template inspired from https://github.com/hleskela/kth-master-thesis-templates/

\documentclass[12pt]{article}

% This is the preamble, load any packages you're going to use here
\usepackage{enumerate} % allows us to customize our lists
\usepackage{pgfgantt} % for gantt chart
\usepackage{geometry} % to change margins
\usepackage{ragged2e} % provides \RaggedLeft
\usepackage{hyperref} % provides \href

\begin{document}

\title{Specification and schedule}
\author{David Wobrock\ \texttt{wobrock@kth.se}}

\maketitle

\section*{Formalities}

\begin{itemize}
 \item Preliminary title: Spectral Graph Theory and High Performance Computing
 \item CSC Supervisor: Stefano Markidis (\texttt{markidis@kth.se})
 \item CSC Examiner: Erwin Laure (\texttt{erwinl@pdc.kth.se})
 \item Supervisor at workplace: Fr\'{e}d\'{e}ric Nataf (\texttt{nataf@ann.jussieu.fr})
\end{itemize}

\section*{Background \& Objective}

\subsection*{Description of the area within which the degree project is being carried out}

Spectral Graph Theory is a developing field with many applications through sociology, chemistry, biology, electrical engineering, computer science, topology and social networks such as Facebook or LinkedIn.
Network science studies the mathematical structure of the graphs that arise in these diverse fields, and the design, analysis, and applications of algorithms that compute with and on them.
For now days large graphs as the ones from social networks, memory requirements and computational times have become an issue.

Recent talks and research, especially at Google, have given that this graph theory can also be applied for new and fast image processing methods.

\subsection*{The principal's interest}

After working a lot with second order derivatives, domain decomposition and high performance computing, it is an interesting topic to explore more possibilities of spectral graph theory.
This is an important reason why delving into image processing is an application of interest.

\subsection*{Objective}

The objective of this project is to explore the usage of graph laplacian operator on images to compute filters, do segmentation, etc.
After applying these techniques on 2-D images, they will also be used on 3-D models.

\section*{Research Question \& Method}

\subsection*{The question that will be examined}

How to adapt the partial derivatives methods for matrices coming from graphs.

\subsection*{Specified problem definition}

Images can be interpreted as graphs, for which the Laplacian matrix can be used to build image processing tools.
These tools have been used for 2-D images so far, but not yet for 3-D models, which will be one of the biggest challenges of this degree project.

\subsection*{Examination method}

Smoothing, deblurring and sharpening 2-D images are the first tasks.
They imply using approximation algorithms for the affinity matrix of the image graph, such as the Nystr\"om extension.
To build different filters, it will be necessary to use different kernel functions, but also try out different sampling techniques to approximate the image.

When the results are satisfying, we will explore the same techniques on 3-D images.

\subsection*{Expected scientific results}

We first expect to master some image processing applications for 2-D images and have them work properly and efficiently in order to apply the technique to 3-D models.
We expect that the same techniques can be used for 3-D models, by extending the current theorems, and have fast 3-D model processing tools when using HPC.

\section*{Evaluation \& News Value}

\subsection*{Evaluation}

The degree project has been fulfilled once we achieve 3-D models processing in a reasonable time.
2-D images can be efficiently processed on smartphones, which gives an idea on the achievable goals for 3-D image processing.

\subsection*{The work's innovation/news value}

The added value of this project is new and fast processing techniques for 3-D models.

\section*{Pilot Study}

\subsection*{Description of the literature studies}

The literature study will focus firstly on spectral graph theory and graph laplacian operator.

\href{https://www.pathlms.com/siam/courses/2426/sections/3234}{This talk} will serve as base for the pre-study, including its references.

\section*{Conditions \& Schedule}

List of resources: \\
\begin{itemize}
 \item hardware: the supplied laptop, servers for more heavy computations
 \item software: Python (Numpy), \LaTeX
 \item people: an available supervisor and friends
 \item perseverance: because it's tough to write a master thesis
\end{itemize}

\subsection*{Collaboration with the principal}

The collaboration with the supervisor happens at least once a week with a meeting.
More meetings are to be expected if particular help is needed at some point.
The whole research team is meeting once every two weeks.

\section*{Schedule}

As an early planning of the degree project, it is expected that the first part of understanding and implementing what exists for 2-D images can take put to two months.
Then three months are planned to apply this to 3-D images.
And finally one month for writing and completing the report and presentation.

\end{document}
