\documentclass{beamer}

\setbeamercolor{footline}{fg=blue}
\setbeamerfont{footline}{series=\bfseries}
\addtobeamertemplate{navigation symbols}{}{
    \usebeamerfont{footline}
    \usebeamercolor[fg]{footline}
    \hspace{2em}
    \insertframenumber/\inserttotalframenumber
}

\title{Image Processing using Graph Laplacian Operator}
\author{David Wobrock \\ \texttt{david.wobrock@gmail.com}}
\institute{ALPINES Team - INRIA Paris \\ KTH, Stockholm \\ INSA, Lyon}
\date{\today}
\subject{Computer Science and Applied Mathematics}

\begin{document}

\frame{\titlepage}

\begin{frame}
 \frametitle{Table of Contents}
 \tableofcontents
\end{frame}

\section[Section]{Introduction}

\begin{frame}
 \frametitle{Background}
 \begin{itemize}
  \item Large-scale application - millions of pictures processed by Google daily
  \item Image processing using spectral graph theory
  \item Involves linear algebra and solving linear systems
  \item Opportunity for high-performance computing and parallelism on dense matrix operations
 \end{itemize}
\end{frame}

\begin{frame}
 \frametitle{Objective}
 \begin{itemize}
  \item Not necessarily improving image processing
  \item Analyse the behaviour of solving large dense systems
 \end{itemize}
 \begin{itemize}
  \item Large: \(N^2\), \(N\) the number of pixels in the input pixels of image
  \item Dense: affinity and Laplacian matrices from image
 \end{itemize}
\end{frame}

\section[Section]{Image processing using Laplacian operator}

\begin{frame}
 \frametitle{Image processing}
\end{frame}

\section[Section]{Parallel implementation}

\begin{frame}
 \frametitle{Implementation}
\end{frame}

\section[Section]{Conclusion}

\begin{frame}
 \frametitle{Conclusion}
\end{frame}

\end{document}
