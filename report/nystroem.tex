\documentclass[]{article}

\usepackage[backend=bibtex, natbib=true, style=numeric, sorting=none]{biblatex}
\addbibresource{bib.bib}

\usepackage{amsmath}
\usepackage{amsfonts}
\usepackage{dsfont}

\DeclareMathOperator{\Lapl}{\mathcal{L}}

\title{Nystr\"om extension}
\author{David Wobrock\\ \texttt{david.wobrock@inria.fr}}

\begin{document}

\maketitle

In a first phase, we would want to compute the largest eigenvalues of \(K\).
However, \(K\) is far too big to be held in memory.
Instead, let's use the Nystr\"om extension which involves computing the eigenvalues \(K_{AA}\) and approximating the largest eigenvectors of \(K\).

\cite{glide_2014} defines the approximation \(\tilde{K}\) of \(K\) as:
\[\tilde{K} = \begin{bmatrix}
 K_{AA} & K_{AB} \\
 K_{AB}^T & K_{AB}^T K_{AA}^{-1} K_{AB}
\end{bmatrix}.\]
With \(K_{AA}\) of size \(p \times p\) and \(p \ll N\) with \(N\) the number of pixels.
So the huge submatrix \(K_{BB}\) is approximated by \(K_{BB} \approx K_{AB}^T K_{AA}^{-1} K_{AB}\).
And let \(\tilde{D}\) be a diagonal matrix such as \(\tilde{D} = diag(\tilde{K} \mathds1)\).

\paragraph{System of linear equations}
\[\begin{pmatrix}
 K_{AA} & K_{AB} \\
 K_{AB}^T & K_{AB}^T K_{AA}^{-1} K_{AB}
\end{pmatrix}
\begin{pmatrix}
 v_A \\
 v_B
\end{pmatrix} = 
\begin{pmatrix}
 b_A \\
 b_B
\end{pmatrix}.\]

From the first line of the system
\begin{equation}
 \begin{split}
   b_A & = K_{AA} v_A + K_{AB} v_B \\
   K_{AA} v_A & = b_A - K_{AB} v_B \\
   v_A & = K_{AA}^{-1}(b_A - K_{AB} v_B)
 \end{split}
\end{equation}
All solution will have this form.

The second line of the linear system
\begin{equation}
 \begin{split}
  b_B & = K_{AB}^T v_A + K_{AB}^T K_{AA}^{-1} K_{AB} v_B \\
  & = K_{AB}^T K_{AA}^{-1}(b_A - K_{AB} v_B) + K_{AB}^T K_{AA}^{-1} K_{AB} v_B \\
  & = K_{AB}^T K_{AA}^{-1} b_A - K_{AB}^T K_{AA}^{-1} K_{AB} v_B + K_{AB}^T K_{AA}^{-1} K_{AB} v_B \\
  & = K_{AB}^T K_{AA}^{-1} b_A
 \end{split}
\end{equation}
Thus if \(b_B = K_{AB}^T K_{AA}^{-1} b_A\) then a solution exists, but it is not unique.

So the image of the matrix:
\[Im(\tilde{K}) = \begin{pmatrix}
 b_A \\
 K_{AB}^T K_{AA}^{-1} b_A
\end{pmatrix}.\]

Given the Nystr\"om extension \cite{fowlkes_spectral_2004} and \cite{glide_2014} \(\tilde{\Phi} = \begin{bmatrix}\Phi_A \\ K_{AB}^T \Phi_A \Pi_A^{-1}\end{bmatrix}\):
\begin{equation}
 \begin{split}
  K_{AA} & = \Phi_A \Pi_A \Phi_A^T \\
  K_{AA}^{-1} & = \Phi_A \Pi_A^{-1} \Phi_A^T \\
  K_{AA}^{-1} \Phi_A & = \Phi_A \Pi_A^{-1} \Phi_A^T \Phi_A \\
  K_{AA}^{-1} \Phi_A & = \Phi_A \Pi_A^{-1}
 \end{split}
\end{equation}
So, as \(\tilde{\Phi} \in Im(\tilde{K})\), we can say that the largest eigenvectors
\begin{equation}
 \begin{split}
  \tilde{\Phi} & = \begin{bmatrix}\Phi_A \\ K_{AB}^T \Phi_A \Pi_A^{-1}\end{bmatrix} \\
               & = \begin{bmatrix}\Phi_A \\ K_{AB}^T K_{AA}^{-1} \Phi_A\end{bmatrix} \\
 \end{split}
\end{equation}

\paragraph{Filter approximation}

Let the filter be defined from the re-normalised Laplacian definition
\begin{equation}
 \begin{split}
  W & = I  - \frac{1}{\bar{d}} (D - K) \\
    & = \begin{bmatrix}
         W_A & W_{AB} \\
         W_{AB}^T & W_B
        \end{bmatrix}
 \end{split}
\end{equation}
We want the largest eigenvalues of this filter.
They correspond to the smallest eigenvalues of \(\Lapl = \frac{1}{\bar{d}} (D - K)\).
Still, holding these matrices in memory is not feasable with respect to the resolution of nowaday pictures.

Let's instead, compute the eigenvalues of \(W_A = I - \frac{1}{\bar{d}} (D_A - K_A\) and extend them using Nystr\"om: \(\begin{bmatrix}V_A \\ W_{AB}^T W_{AA}^{-1} V_A\end{bmatrix}\).

We want the smallest eigenvalue \(\tilde{\mu}\) of \(\tilde{\Lapl}\):
\begin{equation}
 \begin{split}
  \tilde{\mu} & = min \frac{\left\langle \tilde{\Lapl} \begin{pmatrix}b_A \\ K_{AB}^T K_{AA}^{-1} b_A\end{pmatrix}, \begin{pmatrix}b_A \\ K_{AB}^T K_{AA}^{-1} b_A\end{pmatrix} \right\rangle}{\left\| \begin{pmatrix}b_A \\ K_{AB}^T K_{AA}^{-1} b_A\end{pmatrix} \right\|} \\
             & = min \frac{\left\langle \tilde{\Lapl} \begin{pmatrix}b_A \\ K_{AB}^T K_{AA}^{-1} b_A\end{pmatrix}, \begin{pmatrix}b_A \\ K_{AB}^T K_{AA}^{-1} b_A\end{pmatrix} \right\rangle}{\langle b_A, b_A \rangle + \langle K_{AB}^T K_{AA}^{-1} b_A, K_{AB}^T K_{AA}^{-1} b_A \rangle}
 \end{split}
\end{equation}

\(\tilde{\Lapl} = \frac{1}{\bar{\tilde{d}}} \tilde{D} - \tilde{K}\). The system of linear equations:

\[\begin{pmatrix}
 \tilde{D_A} - K_{AA} & -K_{AB} \\
 -K_{AB}^T & \tilde{D_C} - K_{AB}^T K_{AA}^{-1} K_{AB}
\end{pmatrix}
\begin{pmatrix}
 w_A \\
 w_B
\end{pmatrix} = 
\begin{pmatrix}
 c_A \\
 c_B
\end{pmatrix}.\]

Again, from the first line:
\begin{equation}
 \begin{split}
  c_A & = (\tilde{D_A} - K_{AA}) w_A - K_{AB} w_B \\
  (\tilde{D_A} - K_{AA}) w_A & = c_A + K_{AB} w_B \\
  w_A & = (\tilde{D_A} - K_{AA})^{-1} (c_A + K_{AB} w_B)
 \end{split}
\end{equation}

And with the second line:
\begin{equation}
 \begin{split}
  c_B & = -K_{AB}^T w_A + (\tilde{D_C} - K_{AB}^T K_{AA}^{-1} K_{AB}) w_B \\
      & = -K_{AB}^T (\tilde{D_A} - K_{AA})^{-1} (c_A + K_{AB} w_B) + (\tilde{D_C} - K_{AB}^T K_{AA}^{-1} K_{AB}) w_B \\
 \end{split}
\end{equation}

With \(\tilde{\Lapl_A} = \tilde{D_A} - K_{AA}\) and \(\tilde{\Lapl_C} = \tilde{D_C} - K_{AB}^T K_{AA}^{-1} K_{AB}\) we write in a more readable manner
\begin{equation}
 \begin{split}
  c_B & = \tilde{\Lapl_C} w_B - K_{AB}^T \tilde{\Lapl_A}^{-1} (c_A + K_{AB} w_B) \\
%      & = \tilde{\Lapl_C} w_B - K_{AB}^T \tilde{\Lapl_A}^{-1} c_A - K_{AB}^T \tilde{\Lapl_A}^{-1} K_{AB} w_B \\
%      & = (\tilde{\Lapl_C} - K_{AB}^T \tilde{\Lapl_A}^{-1} (c_A w_B^{-1} + K_{AB})) w_B \\
 \end{split}
\end{equation}

\clearpage
\printbibliography

\end{document}
