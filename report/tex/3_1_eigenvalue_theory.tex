\section{Theory}

\paragraph{}
The filter \(W\) is built on top of the kernel matrix \(K\) measuring the similarity between each pixel.
The most popular kernel functions are the \textit{Bilateral filter} \cite{bilateral_tomasi_1998} and the \textit{Non-local Mean filter} \cite{kervrann_nlm_2006}.
In general, the kernel functions create a symmetric positive semi-definite (PSD) matrix $K$ with \(k_{ij} \ge 0\).

\paragraph{}
In our case, we shall use the re-normalised Laplacian \cite{siam_slides_2016}, which will result in a normalisation-free filter \cite{milanfar_new_2016}.
We define the Laplacian operator as
\[\Lapl = \alpha (D - K),\]
with \(alpha = \bigO (\bar{d}^{-1})\), \(\bar{d} = mean(d_i)\), \(d_i = \sum^n_{j=1} K_{ij}\) and \(D = diag(d_i)\).
For its definition, we know that \(\Lapl\) is symmetric and its eigenvalues \(0 \le \mu_i \le 1\), meaning that \(\Lapl\) is symmetric positive definite (SPD).

\paragraph{}
The filter is defined as \(W = I - \Lapl\). The identity \(I\) is obviously SPD, so the filter is also SPD.
We know from \cite{glide_2014} that the eigenvalues of \(W\) are defined as \(0 \le \lambda^W_i \le 1\) and the largest eigenvalue \(\lambda^W_1 = 1\).

\paragraph{}
The image processing algorithm contains the computation of the eigendecomposition of the submatrix \(W_A\).
From the properties of SPD matrices, since \(W_A\) is a principal submatrix of \(W\), is it also SPD.
Furthermore, we can say that the eigenvalues \(0 \le \lambda^{W_A}_i \le 1\) and \(\lambda^{W_A}_1 \le 1\).

\paragraph{Proof}
Let \(A\) be a symmetric matrix of size \(n\), \(\lambda^A_n\) be the largest eigenvalue of \(A\) and \(\lambda^A_1\) the smallest one.
Let \(R\) be the restriction operator, such as, with \(u\) a non-zero vector, \(Ru = \begin{pmatrix}\alpha_1 \\ \alpha_2 \\ \vdots \\ 0 \\ \vdots \\ 0 \end{pmatrix}\) for example.
This defines \(RAR^T\) a \(s \times s\) principal submatrix of \(A\) with  \(s \in [1; n]\).
Suppose the remaining rows and columns of \(A\) in \(RAR^T\) are indexed by \(S\) of size \(s\). \\
Let \(\mathcal{U} \in \Real^s\) and \(u \in \Real^n\) with \(\begin{cases} u_i = \mathcal{U}_i & \quad \text{if } i \in S \\ u_i = 0 & \quad \text{if } i \notin S \end{cases}\).
Given a \(k \in [1; s]\), the Courant-Fischer theorem, involving the Rayleigh-Ritz quotient, implies that,
\[max\left(\frac{\langle Au, u \rangle}{\langle u, u\rangle}\right) = max\left(\frac{\langle RAR^T\mathcal{U}, \mathcal{U}\rangle}{\langle \mathcal{U}, \mathcal{U} \rangle}\right) \le \lambda^A_k.\]
So \(\lambda^{RAR^T}_k \le \lambda^A_k\), thus \(\lambda^{RAR^T}_k \le \lambda^A_n\) and finally \(\lambda^{RAR^T}_s \le \lambda^A_n\).
This concludes the proof, showing that the eigenvalues of the submatrix are bounded by the eigenvalues of the original matrix.

\paragraph{}
From the definition of the filter \(W = I - \Lapl\), we have the submatrix \(W_A = I - \Lapl_A\), with \(I\) being the identity of appropriate order.
For the algorithm, we need to compute the largest eigenvalues of \(W_A\).

\paragraph{Theorem}
Computing the largest eigenvalues of \(W_A\) is equivalent to computing the smallest eigenvalues of \(\Lapl_A\).

\paragraph{Proof}

\begin{equation}
 \begin{split}
     W_A x = \lambda x & \Leftrightarrow (I - \Lapl_A)x = \lambda x \\
                     & \Leftrightarrow x - \Lapl_A x = \lambda x \\
                     & \Leftrightarrow \Lapl_A x = x - \lambda x \\
                     & \Leftrightarrow \Lapl_A x = (1 - \lambda) x
 \end{split}
\end{equation}
So the eigenvalues of the Laplacian submatrix \(\mu = 1 - \lambda\).
We know that \(\mu \ge 0\), so \(1 - \lambda \ge 0\).

We can thus get the greatest eigenvalues of \(W_A\) by computing the smallest eigenvalues of \(\Lapl_A\).

\paragraph{Speed of convergence}
For both these problems, finding the greatest and smallest eigenvalues, the most famous methods are, respectively, the power method and inverse power method.

For the power iteration, the convergence rate is \(|\frac{\lambda_2}{\lambda_1}|\), with \(\lambda_2\) being the second largest eigenvalue.
We know that \(\lambda^{W_A}_2 \le \lambda^{W_A}_1 \le 1\) and thus \(\frac{\lambda^{W_A}_2}{\lambda^{W_A}_1} \le \frac{\lambda^{W_A}_1}{\lambda^{W_A}_1} = 1\).
The convergence rate is lower than 1.

The inverse iteration has a speed of convergence of \(\frac{\mu_1}{\mu_2}\), with \(\mu_2\) the second smallest eigenvalue.
Again, we know that \(0 \le \mu^{\Lapl_A}_1 \le \mu^{\Lapl_A}_2\).
