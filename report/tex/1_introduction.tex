\chapter{Introduction}

\section{Background}

The talk by Milanfar \cite{siam_slides_2016}, working at Google Research, about using the Graph Laplacian Operator for Image Processing purposes awakes curiosity.

Indeed, Milanfar reports that these techniques to build filters are used on smartphones, which means it is done in a reasonable time with limited computational resources.
Over 2 billion photos are shared daily on social media \cite{siam_slides_2016}, with very high resolutions and filters are often applied on them.

% Other filters problem, why we need this report and work here

\section{Objective}

The objective of this thesis is to, in a first place, understand and implement some image processing algorithms shown by Milanfar's work on 2D images.
We want to observe the performance and results of these algorithms to be validate the usage of intrinsic spectral properties of an image.
For instance, the spectrum of the graph associated to an image can serve for image segmentation.
Applied in a fine-grained manner, this segmentation could lead to image compression.
More classic algorithm can also be achieved with the described approach, such as image denoising, deblurring, sharpening, etc.

The next step includes extending these techniques on 3D images.
Given the results, new scalable processing approaches for 3D images could be explored.
These also include standard processing filters such as denoising and sharpening, but also segmentation and compression techniques.

These approaches, even though faster than other filters, still require computations which can grow fast in the 3D case.
That is why we want to apply the algorithm on high performance architectures in order to achieve reasonable computation times.

\section{Related work}

\subsection{Denoising}

\paragraph{Background}

Even with high quality cameras, denoising and improving the taken picture remains important.
The two main issues that have to be addressed by denoising are blur and noise.
The effect of blur is internal to cameras since the number of samples of the signal and it has to hold the Shannon-Nyquist theorem \cite{buades_review_2005}, and noise comes from the light aquisition system that fluctuates in relation to the amount of incoming photons.

To model these problems, we can formulate the corrupted image as,
\[y = z + e,\]
where \(e\) is the noise vector of variance \(\sigma^2\), \(z\) the clean signal vector and \(y\) the noisy picture.

What we want is a high-performance denoiser, capable of scaling up in relation to increasing the image size and keeping reasonable performances.
The output image should come as close as possible to the clean image.
As an important matter, it is now accepted that images contain a lot of redundancy.
This means that, in a natural image, every small enough window has many similar windows in the same image.

\paragraph{Traditional, patch-based methods}

The image denoising algorithms review proposed by \cite{buades_review_2005} suggests that the NL-means algorithm, compared to other reviewed methods, comes closest to the original image when applied to a noisy image.
This algorithm takes advantage of the redundancy of natural images and for a given pixel i, we can predict its value by using the pixels in its neighbourhood.

In \cite{dabov_image_2007}, the authors propose the BM3D algorithm, a denoising strategy based on grouping similar 2-D fragments of the image into 3-D data arrays. Then, collaborative filtering is performed on these groups and return 2-D estimates of all grouped blocks.
This algorithm exposed state-of-the-art performance in terms of denoising at that time. The results still are one of the best for a reasonable computational cost. 

\paragraph{Global filter}

In the last couple of years, global image denoising filters came up, based on spectral decompositions \cite{glide_2014}.
This approach considers the image as a complete graph, where the filter value of each pixel is approximated by all pixels in the image.
We define the approximated clean image \(\hat{z}\) by,
\[\hat{z} = Wy,\]
where \(W\) is our data-dependent global filter, a \(n \times n\) matrix, \(n\) the number of pixels in the picture.
\(W\) is computed from the graph affinity matrix\footnote{Also called kernel matrix or similarity matrix} \(K\), also of size \(n \times n\), such as
\[K = {\mathcal{K}_{ij}},\]
where \(\mathcal{K}\) is a similarity function between two pixels \(i\) and \(j\).
As the size of \(K\) can grow very large, we sample the image and compute an approximated \(\hat{K}\) on this subset, using the Nystr\"om extension.
\(K\) can in fact be approximated through its eigenvectors.
Knowing that the eigenvalues of \(K\) decay very fast \cite{siam_slides_2016}, the first eigenvectors of the subset of pixels are enough to compute the approximated \(\hat{K}\).

Generally, as proposed in \cite{glide_2014} and \cite{talebi_asymptotic_2016}, to improve the denoising performance of global filters, pre-filtering techniques are used.
It is proposed to first apply a NL-means algorithm to the image to reduce the noise, but to still compute the global filter on the noisy input and apply it to the pre-filtered image.

\paragraph{Comparison between patch-based and global methods}

As \cite{talebi_asymptotic_2016} suggests, global filter methods have the possibility to converge to a perfect reconstruction of the clean image, which seems to be impossible for techniques like BM3D.
Global filtering also seems promising for creating more practical image processing algorithms.

The thesis \cite{kheradmand_graph-based_2016} proposes a normalised iterative denoising algorithm which is patch-based.
The work reports that this technique has slightly better results than the global filter but essentially has a better runtime performance.

\subsection{Sharpening}

\paragraph{Background}

Sharpening consists basically in a high pass filter which will magnify high frequency details \cite{kheradmand_graph-based_2016}.
The two main issues again with this approach are that noise often displays high frequency attributes, which will be amplified by the filter.
Secondly, the concept of overshoot and undershoot, which appears when sharpening already sharp parts (edges for example), will exhibit unpleasant artifacts.

Modeling the problem is the same as for denoising, \(\hat{z} = Fy\), \(F\) being our filter.

\paragraph{Classical Difference of Gaussian (DoG) operator}

This technique consists of computing the difference between two gaussian kernels, which will produce a range of different kernels with various frequencies.
% TODO more + sources

\paragraph{Structure-aware sharpening}

We know from its definition \cite{kheradmand_non-linear_2015} that the smoothing filter \(W\) is a symmetric and doubly stochastic matrix\footnote{Depending on the used Laplacian definition. For example the one obtained by ``Sinkhorn" iterations \cite{milanfar_symmetrizing_2013}}.
So the largest eigenvalue \(\lambda_1 = 1\) shows that it has low pass filter characteristics.
With the definition of the Laplacian, in relation to the matrix \(W\), as \(\Lapl = I - W\), we can consider that it is a data-adaptive high pass filter \cite{kheradmand_graph-based_2016}.
So we can define a data-adaptive unsharp mask filter as
\[F_1 = I + \beta (I-W),\]
with \(\beta > 0\). This is basically a weighted high pass filtered version of the input image \cite{siam_slides_2016}.

But with this approach, the noise amplifying and overshoot problems remain.
The proposed solution in \cite{kheradmand_non-linear_2015} applies first a smoothing filter to the image, then the unsharp mask filter and finally the same smoothing filter again.
The smoothing and sharping filters use possibly two different kernels matrices as basis.
The first smoothing filter aims to reduce the noise, while we still avoid over-smoothing.
The second and final smoothing controls the effect of the amplified noise and the overshoot artifacts.
This gives us the filter
\[F = W_1(I + \beta (I - W_2))W_1,\]
where \(W_1\) and \(W_2\) are constructed from the affinity matrices \(K_1\) and \(K_2\), which can differ in relation to the parameters of their respective kernel function.


\section{Delimitations}

The number of investigated algorithms and variants depend on the pre-study and objectives that make the most sense for 3D images.
It is obviously also limited by the amount of time given to the degree project.
