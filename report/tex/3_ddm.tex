\chapter{Dense Linear Systems and High Performance Computing}

\section{Theory}

\paragraph{}
The filter $W$ is built on top of the kernel matrix $K$ measuring the similarity between each pixel.
The most popular kernel functions are the \textit{Bilateral filter} \cite{bilateral_tomasi_1998} and the \textit{Non-local Mean filter} \cite{kervrann_nlm_2006}.
Both of those create a symmetric positive semi-definite matrix such as $k_{ij} \ge 0$.


\section{Motivation and tools}

\paragraph{}
In the case of our image processing algorithm, choosing a sufficient number of sample pixels is essential for a good approximation of the matrices eigenvalues and eigenvectors.
As exposed in the litterature \cite{glide_2014} \cite{fowlkes_spectral_2004}, less than 1\% of the pixels seems to be enough to capture most of the image information.
However, applied to very high resolution images, 1\% of the number of pixels is still a large amount and causes to compute large dense matrices.
Additionally, we intend to apply this algorithm on 3D images where the problem size grows even further.

\paragraph{}
As exposed previously, the algorithm contains matrix computations and eigenvalue problems.
The computations are mostly independent (row independent for most matrix computations).
Therefore, there is an opportunity and a need to speed up the algorithm by parallalising it on supercomputers.
For this, we use the PETSc library (Portable, Extensible Toolkit for Scientific Computation) \cite{petsc_web_page}, which makes use of HPC tools like the MPI standard to distribute and compute efficiently matrices and vectors.
Furthermore, the library SLEPc (Scalable Library for Eigenvalue Problem Computations) \cite{hernandez_2005_slepc}, based on PETSc, is used to solve the eigenvalue problems efficiently.

\section{Experimental results}

\paragraph{Architecture precisions}

\paragraph{Algorithm precision}

\paragraph{Observed runtimes}

\paragraph{Image results}
