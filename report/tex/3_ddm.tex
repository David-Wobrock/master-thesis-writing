\chapter{Linear Systems and Domain Decomposition Methods}

\section{Theory}

\paragraph{}
The filter \(W\) is built on top of the kernel matrix \(K\) measuring the similarity between each pixel.
The most popular kernel functions are the \textit{Bilateral filter} \cite{bilateral_tomasi_1998} and the \textit{Non-local Mean filter} \cite{kervrann_nlm_2006} to measure these similarities.
The kernel functions create a symmetric positive semi-definite (PSD) matrix \(K\), so the eigenvalues of \(K\) are non-negative, \(\lambda_K \ge 0\), as described in \cite{talebi_fast_2016}.
Also, from the definition of the filters, \(\forall i, j \in [1, N]\), \(N\) the number of pixels, the values of the affinity matrix \(K\) are non-negative \(K_{ij} \ge 0\).

\paragraph{}
In our case, we shall use the re-normalised Laplacian \cite{siam_slides_2016}, which will result in a normalisation-free filter.
\cite{siam_slides_2016} and \cite{milanfar_new_2016} define this Laplacian operator as
\[\Lapl = \alpha (D - K),\]
with \(alpha = \bigO (\bar{d}^{-1})\), \(\bar{d} = \sum^N_{i=1} \frac{d_i}{N}\), \(d_i = \sum^N_{j=1} K_{ij}\) and \(D = diag(d_i)\).
From the definition given by \cite{siam_slides_2016}, \(\Lapl\) is symmetric positive definite (SPD) and its eigenvalues \(0 \le \mu_i \le 1\).

\paragraph{}
The filter is implicitely defined by \cite{modern_tour_2013} as \(W = I - \Lapl\) and \(W\) is SPD as we consider a SPD Laplacian.
We know from \cite{glide_2014} that the eigenvalues of \(W\) are defined as \(0 \le \lambda^W_i \le 1\) and the largest eigenvalue \(\lambda^W_1 = 1\).

\paragraph{}
The image processing algorithm contains the computation of the eigendecomposition of the submatrix \(W_A\).
From the properties of SPD matrices, since \(W_A\) is a principal submatrix of \(W\), is it also SPD.
Furthermore, we can say that the eigenvalues \(0 \le \lambda^{W_A}_i \le 1\) and \(\lambda^{W_A}_1 \le 1\).

\paragraph{Proof}
Let \(A\) be a symmetric matrix of size \(n\), \(\lambda^A_n\) be the largest eigenvalue of \(A\) and \(\lambda^A_1\) the smallest one.
Let \(R\) be the restriction operator, such as, with \(u\) a non-zero vector, \(Ru = \begin{pmatrix}\alpha_1 \\ \alpha_2 \\ \vdots \\ 0 \\ \vdots \\ 0 \end{pmatrix}\) for example.
This defines \(RAR^T\) a \(s \times s\) principal submatrix of \(A\) with  \(s \in [1; n]\).
Suppose the remaining rows and columns of \(A\) in \(RAR^T\) are indexed by \(S\) of size \(s\). \\
Let \(\mathcal{U} \in \Real^s\) and \(u \in \Real^n\) with \(\begin{cases} u_i = \mathcal{U}_i & \quad \text{if } i \in S \\ u_i = 0 & \quad \text{if } i \notin S \end{cases}\).
Given a \(k \in [1; s]\), the Courant-Fischer theorem, involving the Rayleigh-Ritz quotient, implies that,
\[max\left(\frac{\langle Au, u \rangle}{\langle u, u\rangle}\right) = max\left(\frac{\langle RAR^T\mathcal{U}, \mathcal{U}\rangle}{\langle \mathcal{U}, \mathcal{U} \rangle}\right) \ge \lambda^A_k.\]
So \(\lambda^{RAR^T}_k \ge \lambda^A_k\).
More over, in the other way, we get
\[min\left(\frac{\langle Au, u \rangle}{\langle u, u\rangle}\right) = min\left(\frac{\langle RAR^T\mathcal{U}, \mathcal{U}\rangle}{\langle \mathcal{U}, \mathcal{U} \rangle}\right) \le \lambda^A_{k+n-s}.\]
And so again, \(\lambda^{RAR^T}_k \le \lambda^A_{k+n-s}\).
This concludes the proof, showing that the eigenvalues of the submatrix are bounded by the eigenvalues of the original matrix.
More precisely, we proved the interlacing property of the eigenvalues of \(RAR^T\) such as
\[\lambda^A_k \le \lambda^{RAR^T}_k \le \lambda^A_{k+n-s}.\]

\paragraph{}
From the definition of the filter \(W = I - \Lapl\), we have the submatrix \(W_A = I - \Lapl_A\), with \(I\) being the identity of appropriate order.
For the algorithm, we need to compute the largest eigenvalues of \(W_A\).

\paragraph{Theorem}
Computing the largest eigenvalues of \(W_A\) is equivalent to computing the smallest eigenvalues of \(\Lapl_A\).

\paragraph{Proof}

\begin{equation}
 \begin{split}
     W_A x = \lambda x & \Leftrightarrow (I - \Lapl_A)x = \lambda x \\
                     & \Leftrightarrow x - \Lapl_A x = \lambda x \\
                     & \Leftrightarrow \Lapl_A x = x - \lambda x \\
                     & \Leftrightarrow \Lapl_A x = (1 - \lambda) x
 \end{split}
\end{equation}
So the eigenvalues of the Laplacian submatrix \(\mu = 1 - \lambda\).
We know that \(\mu \ge 0\), so \(1 - \lambda \ge 0\).

We can thus get the greatest eigenvalues of \(W_A\) by computing the smallest eigenvalues of \(\Lapl_A\).

\paragraph{Speed of convergence}
For both these problems, finding the greatest and smallest eigenvalues, the most famous methods are, respectively, the power method and inverse power method\footnote{Those are also called power iteration and inverse iteration. The inverse method has a variant called inverse subspace iteration, to find the associated subspace to the eigenvalues.}.

For the power iteration, the convergence rate is \(|\frac{\lambda_2}{\lambda_1}|\), with \(\lambda_2\) being the second largest eigenvalue.
We know that \(\lambda^{W_A}_2 \le \lambda^{W_A}_1 \le 1\) and thus \(\frac{\lambda^{W_A}_2}{\lambda^{W_A}_1} \le \frac{\lambda^{W_A}_1}{\lambda^{W_A}_1} = 1\).
The convergence rate is lower than 1.
The method is fast if the rate is small and slow if the rate is close to 1.
So the closer the two eigenvalues are, the slower the method converges.

The inverse iteration has a speed of convergence of \(|\frac{\mu_1}{\mu_2}|\), with \(\mu_2\) the second smallest eigenvalue.
Again, we know that \(0 \le \mu^{\Lapl_A}_1 \le \mu^{\Lapl_A}_2\).
So the convergence speed is also lower than 1.

We come to the conclusion that both methods depend on the spacing between the eigenvalues.
The closer they are, the more iterations will be required to converge.
The difference of convergence speeds for both methods therefore depends on the distance between the largest eigenvalues and the distance between the smallest ones.

\paragraph{}
Inverse iterations implies either to compute the inverse of the matrix \(x_{k+1} = A^{-1}x_k\), or to solve a system of linear equations \(Ax_{k+1} = x_k\).
Since the image processing context suggests having dense matrices, we want to explore the performances of Krylov methods and domain decomposition methods (e.g. the Additive Schwarz method) on such dense matrices.


\section{Theoretical basis}

\paragraph{}
Solving a system of linear equations such that
\[Ax = b,\]
is often critical in scientific computing.
When discretising equations coming from physics for example, a huge linear system can be obtained.
Multiple methods exist to solve such systems, even when the system is large and expensive to compute.
We present in the following the most used and known solvers.

\subsection{Direct solvers}

\paragraph{}
The most commonly used solvers for systems of linear equations are direct solvers.
They provide robust methods and optimal solutions to the problem.
However, they can be hard to parallelise and have difficulties with large input.
The most famous is the backslash operator from MATLAB which performs tests to determine which special case algorithm to use, but ultimately falls back on a LU factorisation \cite{mldivide_matlab}.
The LU factorisation, closely related to Gaussian elimination, is hard to parallelise.
A block version of the LU factorisation exists that can be parallelised.
Other direct solvers, like MUMPS, exist and can be used, but generally they reach a computational limit above \(10^6\) degrees of freedom in a 2D problem, and \(10^5\) in 3D.

\subsection{Iterative solvers}

\paragraph{}
For large problems, iterative methods must be used to achieve a reasonable running time.
The two types of iterative solvers are fixed-point iteration methods and Krylov type methods.
Both require only a small amount of memory and can often be parallelised.
The main drawback is that these methods tend to be less robust than direct solvers and convergence depends on the problem.
Indeed, ill-conditioned input matrices will be difficult to solve correctly by iterative methods.
The most relevant iterative methods are the conjugate gradient and GMRES \cite{saad_gmres_1986}.

To tackle the ill-conditioned matrices problem, there is a need to precondition the system.

\subsection{Domain decomposition methods}

\paragraph{}
One of the ways to precondition systems of linear equations is to use domain decomposition.
The idea goes back to Schwarz who wanted to solve a Poisson problem on a complex geometry.
He decomposed the geometry into multiple smaller simple geometric forms, making it easy to work on subproblems.
This idea has been extended and improved to propose fixed-point iterations solvers for linear systems.
However, Krylov methods expose better results and faster convergence, but domain decomposition methods can actually be used as preconditioners to the system.
The most famous Schwarz preconditioners are the Restricted Additive Schwarz (RAS) and Additive Schwarz Method (ASM).
For example, the formulation of the ASM preconditioning matrix
\[M^{-1}_{ASM} = \sum_i R_i^T A_i^{-1} R_i,\]
with \(i\) subdomains and \(R_i\) the restriction matrix of \(A\) to the \(i\)-th subdomain.
With such a preconditioner we will be able to solve
\[M^{-1}Ax = M^{-1}b\]
which exposes the same solution as the original problem.

\section{Motivation and tools}

\paragraph{}
In the case of our image processing algorithm, choosing a sufficient number of sample pixels is essential for a good approximation of the matrices eigenvalues and eigenvectors.
As exposed in the litterature \cite{glide_2014} \cite{fowlkes_spectral_2004}, less than 1\% of the pixels seems to be enough to capture most of the image information.
However, applied to very high resolution images, 1\% of the number of pixels is still a large amount and causes to compute large dense matrices.
Additionally, we intend to apply this algorithm on 3D images where the problem size grows even further.

\paragraph{}
As exposed previously, the algorithm contains matrix computations and eigenvalue problems.
The computations are mostly independent (row independent for most matrix computations).
Therefore, there is an opportunity and a need to speed up the algorithm by parallalising it on supercomputers.
For this, we use the PETSc library (Portable, Extensible Toolkit for Scientific Computation) \cite{petsc_web_page}, which makes use of HPC tools like the MPI standard to distribute and compute efficiently matrices and vectors.
Furthermore, the library SLEPc (Scalable Library for Eigenvalue Problem Computations) \cite{hernandez_2005_slepc}, based on PETSc, is used to solve the eigenvalue problems efficiently.
